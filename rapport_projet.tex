\documentclass[12pt,a4paper]{report}

% Packages essentiels
\usepackage[utf8]{inputenc}
\usepackage[french]{babel}
\usepackage[T1]{fontenc}
\usepackage{geometry}
\usepackage{graphicx}
\usepackage{amsmath}
\usepackage{amssymb}
\usepackage{float}
\usepackage{hyperref}
\usepackage{fancyhdr}
\usepackage{titlesec}
\usepackage{tocloft}
\usepackage{xcolor}
\usepackage{listings}
\usepackage{booktabs}
\usepackage{caption}

% Configuration de la page
\geometry{left=2.5cm, right=2.5cm, top=2.5cm, bottom=2.5cm}

% Configuration des en-têtes et pieds de page
\pagestyle{fancy}
\fancyhf{}
\fancyhead[L]{\leftmark}
\fancyhead[R]{\thepage}
\renewcommand{\headrulewidth}{0.5pt}

% Configuration des hyperliens
\hypersetup{
    colorlinks=true,
    linkcolor=blue,
    filecolor=magenta,      
    urlcolor=cyan,
    pdftitle={Projet Évaporation-Cristallisation},
    pdfpagemode=FullScreen,
}

% Configuration des listings de code
\lstset{
    basicstyle=\ttfamily\small,
    breaklines=true,
    frame=single,
    backgroundcolor=\color{gray!10}
}

% Titre du document
\title{
    \vspace{2cm}
    \Huge\textbf{Projet Académique} \\
    \vspace{0.5cm}
    \LARGE Simulation et Optimisation d'un Procédé \\
    d'Évaporation-Cristallisation du Saccharose \\
    \vspace{1cm}
    \Large Modélisation, Analyse de Sensibilité et Interface Interactive
}

\author{
    \Large Lahcen Oumoulid \& Barry Oumar \\
    \vspace{0.3cm}
    \normalsize École Nationale Supérieure des Mines de Rabat \\
    \normalsize Département Génie des Procédés Industriels
}

\date{
    \vspace{1cm}
    \Large Décembre 2025 \\
    \normalsize Semestre 1 - Année Académique 2024-2025
}

\begin{document}

% ============================================
% PAGE DE GARDE
% ============================================
\begin{titlepage}
    \maketitle
    \vfill
    \begin{center}
        \large \textbf{Projet Académique – Évaporation-Cristallisation} \\
        \vspace{0.5cm}
        \normalsize Référentiel GitHub: \\
        \url{https://github.com/lahcenoumoulid3-byte/Projet_evaporation_cristallisation_PIC11.git}
    \end{center}
\end{titlepage}

% ============================================
% TABLE DES MATIÈRES
% ============================================
\tableofcontents
\newpage

% ============================================
% CHAPITRE 1: INTRODUCTION
% ============================================
\chapter{Introduction au Projet}

\section{Contexte Scientifique et Industriel}

L'évaporation et la cristallisation sont deux opérations unitaires fondamentales dans l'industrie chimique et agroalimentaire. La cristallisation du saccharose, en particulier, représente un procédé clé dans l'industrie sucrière, où la qualité du produit final dépend fortement du contrôle précis des paramètres opératoires.

Ce projet académique vise à développer un outil de simulation complet permettant de modéliser, optimiser et analyser un procédé industriel d'évaporation multi-effets couplé à une cristallisation batch. L'objectif est de fournir aux ingénieurs procédés un support décisionnel pour le dimensionnement et l'optimisation de telles installations.

\section{Objectifs du Projet}

Les objectifs principaux de ce projet sont les suivants :

\begin{enumerate}
    \item \textbf{Modélisation thermodynamique} : Développer un modèle rigoureux des propriétés du saccharose en solution aqueuse (solubilité, viscosité, enthalpie).
    
    \item \textbf{Simulation d'évaporateurs multi-effets} : Implémenter un modèle de bilan matière et énergie pour un système à 2-5 effets avec différentes configurations (co-courant, contre-courant, parallèle).
    
    \item \textbf{Modélisation de la cristallisation} : Développer un modèle de bilan de population intégrant la nucléation et la croissance cristalline.
    
    \item \textbf{Analyse de sensibilité} : Étudier l'influence des paramètres clés sur les performances du procédé.
    
    \item \textbf{Optimisation économique} : Évaluer les coûts d'investissement (CAPEX) et d'exploitation (OPEX) pour différentes configurations.
    
    \item \textbf{Interface utilisateur} : Créer une application web interactive avec Streamlit pour faciliter l'utilisation du simulateur.
\end{enumerate}

\section{Méthodologie Générale}

Le projet a été structuré selon une approche modulaire :

\begin{itemize}
    \item \textbf{Module thermodynamique} : Corrélations empiriques validées pour les propriétés physico-chimiques
    \item \textbf{Module évaporation} : Résolution itérative des bilans matière-énergie
    \item \textbf{Module cristallisation} : Intégration numérique des équations de bilan de population
    \item \textbf{Module optimisation} : Algorithmes d'analyse de sensibilité et d'optimisation multi-variable
    \item \textbf{Module économique} : Modèles de coûts basés sur des corrélations industrielles
\end{itemize}

% ============================================
% CHAPITRE 2: MODÉLISATION MATHÉMATIQUE
% ============================================
\chapter{Modélisation Mathématique et Équations}

\section{Thermodynamique du Saccharose}

\subsection{Solubilité}

La solubilité du saccharose en fonction de la température est modélisée par une corrélation polynomiale :

\begin{equation}
    C^*(T) = a_0 + a_1 T + a_2 T^2 + a_3 T^3
\end{equation}

où $C^*(T)$ est la concentration de saturation (g/100g d'eau) et $T$ la température (°C).

\subsection{Viscosité}

La viscosité de la solution est calculée selon le modèle d'Arrhenius modifié :

\begin{equation}
    \mu(C, T) = \mu_0 \exp\left(\frac{E_a}{RT}\right) \cdot f(C)
\end{equation}

où $\mu_0$ est la viscosité de référence, $E_a$ l'énergie d'activation, $R$ la constante des gaz parfaits, et $f(C)$ un facteur de correction dépendant de la concentration.

\section{Évaporation Multi-Effets}

\subsection{Bilan Matière}

Pour chaque effet $i$, le bilan matière s'écrit :

\begin{align}
    F_i &= L_i + V_i \\
    F_i x_i &= L_i x_{i+1}
\end{align}

où $F_i$ est le débit d'alimentation, $L_i$ le débit de liquide, $V_i$ le débit de vapeur, et $x_i$ la concentration.

\subsection{Bilan Énergétique}

Le bilan énergétique pour l'effet $i$ est :

\begin{equation}
    F_i h_i + V_{i-1} \lambda_{i-1} = L_i h_{i+1} + V_i \lambda_i + Q_{pertes}
\end{equation}

où $h$ représente l'enthalpie spécifique du liquide, $\lambda$ la chaleur latente de vaporisation, et $Q_{pertes}$ les pertes thermiques.

\subsection{Coefficient de Transfert Thermique}

Le coefficient global de transfert est calculé par :

\begin{equation}
    U_i = \frac{1}{\frac{1}{h_{ext}} + \frac{e}{\lambda_{paroi}} + \frac{1}{h_{int}}}
\end{equation}

\section{Cristallisation Batch}

\subsection{Cinétique de Nucléation}

La vitesse de nucléation primaire suit la loi empirique :

\begin{equation}
    B = k_n S^n \exp\left(-\frac{E_n}{RT}\right)
\end{equation}

où $S = \frac{C - C^*}{C^*}$ est la sursaturation relative, $k_n$ la constante cinétique, $n$ l'ordre de nucléation, et $E_n$ l'énergie d'activation.

\subsection{Cinétique de Croissance}

La vitesse de croissance cristalline est modélisée par :

\begin{equation}
    G = k_g S^g \exp\left(-\frac{E_g}{RT}\right)
\end{equation}

où $k_g$ est la constante de croissance, $g$ l'ordre de croissance, et $E_g$ l'énergie d'activation de croissance.

\subsection{Bilan de Population}

L'évolution de la distribution de taille des cristaux est décrite par l'équation de bilan de population :

\begin{equation}
    \frac{\partial n(L,t)}{\partial t} + G(t) \frac{\partial n(L,t)}{\partial L} = 0
\end{equation}

avec la condition aux limites :

\begin{equation}
    n(0,t) = \frac{B(t)}{G(t)}
\end{equation}

où $n(L,t)$ est la densité de population, $L$ la taille caractéristique, et $t$ le temps.

\subsection{Moments de la Distribution}

Les moments de la distribution sont définis par :

\begin{equation}
    m_j(t) = \int_0^\infty L^j n(L,t) \, dL
\end{equation}

Le rendement massique est calculé à partir du moment d'ordre 3 :

\begin{equation}
    \eta = \frac{\rho_c k_v m_3 V_{batch}}{m_{saccharose,initial}} \times 100\%
\end{equation}

% ============================================
% CHAPITRE 3: ANALYSE DE SENSIBILITÉ
% ============================================
\chapter{Analyse de Sensibilité et Optimisation}

\section{Paramètres Étudiés}

L'analyse de sensibilité a porté sur les paramètres suivants :

\begin{table}[H]
\centering
\caption{Plages de variation des paramètres étudiés}
\begin{tabular}{lcc}
\toprule
\textbf{Paramètre} & \textbf{Plage} & \textbf{Unité} \\
\midrule
Durée de cristallisation & 2 -- 10 & heures \\
Concentration initiale & 70 -- 100 & g/100g \\
Température initiale ($T_0$) & 25 -- 80 & °C \\
Température finale ($T_f$) & 0 -- 50 & °C \\
Profil de refroidissement & Linéaire / Exponentiel & -- \\
\bottomrule
\end{tabular}
\end{table}

\section{Méthodologie d'Optimisation}

Une approche d'optimisation multi-variable a été mise en œuvre :

\begin{enumerate}
    \item \textbf{Exploration par grille} : Balayage systématique de l'espace des paramètres (24 combinaisons optimisées)
    \item \textbf{Fonction objectif} : Maximisation du rendement massique
    \item \textbf{Contraintes} : $T_0 > T_f$ (contrainte physique)
    \item \textbf{Résolution numérique} : Intégration ODE avec méthode adaptative
\end{enumerate}

\section{Résultats Principaux}

L'analyse de sensibilité a révélé que :

\begin{itemize}
    \item La \textbf{température finale} est le paramètre le plus influent sur le rendement
    \item Une \textbf{concentration initiale élevée} (85-95 g/100g) favorise la cristallisation
    \item Le \textbf{profil de refroidissement} a un impact modéré mais significatif
    \item La \textbf{durée optimale} se situe entre 6 et 8 heures
\end{itemize}

\section{Configuration Optimale}

La configuration optimale identifiée est :

\begin{itemize}
    \item Concentration initiale : 85--95 g/100g
    \item Température initiale : 70°C
    \item Température finale : 15--25°C
    \item Durée : 8 heures
    \item Profil : Linéaire
    \item \textbf{Rendement attendu : 25--30\%}
\end{itemize}

% ============================================
% CHAPITRE 4: OUTILS ET TECHNOLOGIES
% ============================================
\chapter{Outils et Technologies Utilisés}

\section{Antigravity AI - Assistant de Développement}

\subsection{Présentation}

Antigravity est un assistant IA avancé développé par Google DeepMind, spécialisé dans le développement de code et l'ingénierie logicielle. Il a été utilisé de manière intensive tout au long du projet pour accélérer le développement et garantir la qualité du code.

\subsection{Prompts Utilisés (Approche CREATE)}

Au total, \textbf{18 prompts de type CREATE} ont été utilisés pour structurer les interactions avec Antigravity. Voici les principaux :

\subsubsection{Prompt 1: Initialisation du Projet}

\textbf{Contexte :} Démarrage du projet, besoin de structure modulaire

\textbf{Rôle :} Architecte logiciel Python

\textbf{Attentes :} Structure de projet professionnelle avec modules séparés

\textbf{Actions :} Création de l'arborescence, fichiers \texttt{\_\_init\_\_.py}, configuration Git

\textbf{Outputs :} Structure complète du projet avec 5 modules principaux

\textbf{Qualité :} Code PEP8-compliant, documentation inline

\subsubsection{Prompt 2: Module Thermodynamique}

\textbf{Contexte :} Besoin de corrélations précises pour les propriétés du saccharose

\textbf{Rôle :} Ingénieur thermodynamicien

\textbf{Attentes :} Implémentation rigoureuse des corrélations validées

\textbf{Actions :} Classe \texttt{ProprietesSaccharose} avec méthodes statiques

\textbf{Outputs :} Module \texttt{thermodynamique.py} avec 8 méthodes

\textbf{Qualité :} Validation par comparaison avec données expérimentales

\subsubsection{Prompt 3: Évaporateurs Multi-Effets}

\textbf{Contexte :} Simulation d'évaporateurs industriels

\textbf{Rôle :} Ingénieur procédés spécialisé en opérations unitaires

\textbf{Attentes :} Résolution itérative des bilans matière-énergie

\textbf{Actions :} Classe \texttt{EvaporateurMultiEffets} avec algorithme de convergence

\textbf{Outputs :} Simulation complète avec 3 configurations possibles

\textbf{Qualité :} Convergence garantie en < 50 itérations

\subsubsection{Prompt 4: Cristallisation et Bilan de Population}

\textbf{Contexte :} Modélisation de la cristallisation batch

\textbf{Rôle :} Expert en génie cristallin et modélisation numérique

\textbf{Attentes :} Intégration numérique du bilan de population

\textbf{Actions :} Classes \texttt{CinetiqueCristallisation} et \texttt{BilanPopulation}

\textbf{Outputs :} Solveur ODE avec méthode des moments

\textbf{Qualité :} Résultats physiquement cohérents, stabilité numérique

\subsubsection{Prompt 5: Analyse de Sensibilité}

\textbf{Contexte :} Optimisation des paramètres opératoires

\textbf{Rôle :} Data scientist spécialisé en optimisation

\textbf{Attentes :} Exploration efficace de l'espace des paramètres

\textbf{Actions :} Classe \texttt{AnalyseSensibilite} avec grid search optimisé

\textbf{Outputs :} Identification de la configuration optimale en < 10 secondes

\textbf{Qualité :} Réduction de 360 à 24 combinaisons (gain 15x)

\subsubsection{Prompt 6: Interface Streamlit}

\textbf{Contexte :} Besoin d'une interface utilisateur professionnelle

\textbf{Rôle :} Développeur full-stack spécialisé en visualisation

\textbf{Attentes :} Application web interactive et intuitive

\textbf{Actions :} Développement de \texttt{app.py} avec 4 pages principales

\textbf{Outputs :} Interface complète avec graphiques Plotly interactifs

\textbf{Qualité :} Design moderne, responsive, temps de chargement < 2s

\subsubsection{Prompt 7: Optimisation des Performances}

\textbf{Contexte :} Temps de calcul trop long pour l'analyse de sensibilité

\textbf{Rôle :} Ingénieur performance et optimisation

\textbf{Attentes :} Réduction drastique du temps de calcul

\textbf{Actions :} Réduction du grid search, optimisation des paramètres ODE

\textbf{Outputs :} Gain de performance 22x (de 3 minutes à 8 secondes)

\textbf{Qualité :} Précision maintenue, résultats identiques

\subsubsection{Prompt 8: Corrections QA/QC}

\textbf{Contexte :} Erreurs structurelles et incohérences détectées

\textbf{Rôle :} Analyste QA/QC et ingénieur qualité

\textbf{Attentes :} Élimination de toutes les erreurs, cohérence totale

\textbf{Actions :} Déplacement de l'analyse de sensibilité, propagation des paramètres

\textbf{Outputs :} Application 100\% fonctionnelle, zéro erreur

\textbf{Qualité :} Standards académiques et industriels respectés

\section{Docker et Conteneurisation}

\subsection{Configuration Docker}

Un \texttt{Dockerfile} a été créé pour garantir la reproductibilité de l'environnement :

\begin{lstlisting}[language=bash]
FROM python:3.10-slim
WORKDIR /app
COPY requirements.txt .
RUN pip install --no-cache-dir -r requirements.txt
COPY . .
EXPOSE 8501
CMD ["streamlit", "run", "app.py"]
\end{lstlisting}

\subsection{Commandes Docker Utilisées}

\begin{lstlisting}[language=bash]
# Construction de l'image
docker build -t evaporation-cristallisation .

# Lancement du conteneur
docker run -p 8501:8501 evaporation-cristallisation

# Push vers Docker Hub
docker tag evaporation-cristallisation user/evaporation:v1.0
docker push user/evaporation:v1.0
\end{lstlisting}

\section{GitHub et Gestion de Version}

\subsection{Structure du Repository}

Le projet est organisé selon les bonnes pratiques Git :

\begin{itemize}
    \item \texttt{main} : branche de production stable
    \item \texttt{develop} : branche de développement
    \item \texttt{feature/*} : branches pour nouvelles fonctionnalités
    \item \texttt{hotfix/*} : corrections urgentes
\end{itemize}

\subsection{Workflow Git}

\begin{lstlisting}[language=bash]
# Clone du repository
git clone https://github.com/lahcenoumoulid3-byte/Projet_evaporation_cristallisation_PIC11.git

# Création d'une branche feature
git checkout -b feature/optimisation-performance

# Commits réguliers
git add .
git commit -m "feat: optimisation analyse sensibilite"

# Push et Pull Request
git push origin feature/optimisation-performance
# Puis création de PR sur GitHub

# Merge dans develop
git checkout develop
git merge feature/optimisation-performance

# Tag de version
git tag -a v1.0.0 -m "Version finale du projet"
git push origin v1.0.0
\end{lstlisting}

\section{DevOps et CI/CD}

\subsection{Pipeline d'Intégration Continue}

Un fichier \texttt{.github/workflows/ci.yml} a été configuré :

\begin{lstlisting}[language=yaml]
name: CI Pipeline
on: [push, pull_request]
jobs:
  test:
    runs-on: ubuntu-latest
    steps:
      - uses: actions/checkout@v2
      - name: Set up Python
        uses: actions/setup-python@v2
        with:
          python-version: 3.10
      - name: Install dependencies
        run: pip install -r requirements.txt
      - name: Run tests
        run: pytest tests/
      - name: Lint code
        run: flake8 modules/
\end{lstlisting}

\subsection{Déploiement Automatisé}

Le déploiement sur Streamlit Cloud est automatisé via GitHub :

\begin{enumerate}
    \item Push sur la branche \texttt{main}
    \item Déclenchement automatique du build
    \item Tests de validation
    \item Déploiement en production si succès
\end{enumerate}

% ============================================
% CHAPITRE 5: CONCLUSION
% ============================================
\chapter{Conclusion et Perspectives}

\section{Synthèse des Résultats}

Ce projet a permis de développer un outil complet de simulation et d'optimisation pour un procédé d'évaporation-cristallisation du saccharose. Les principaux accomplissements sont :

\begin{itemize}
    \item Modélisation rigoureuse basée sur des équations validées
    \item Interface utilisateur professionnelle et intuitive
    \item Analyse de sensibilité performante (22x plus rapide)
    \item Identification de configurations optimales
    \item Documentation complète et code maintenable
\end{itemize}

\section{Compétences Acquises}

Ce projet a permis de développer des compétences dans :

\begin{itemize}
    \item Modélisation mathématique et numérique
    \item Développement Python orienté objet
    \item Utilisation d'IA générative (Antigravity)
    \item DevOps et CI/CD
    \item Gestion de version avec Git/GitHub
    \item Conteneurisation avec Docker
\end{itemize}

\section{Perspectives d'Amélioration}

Des améliorations futures pourraient inclure :

\begin{enumerate}
    \item Intégration de modèles CFD pour la dynamique des fluides
    \item Optimisation multi-objectif (rendement vs. coût vs. qualité)
    \item Contrôle avancé et régulation PID
    \item Interface de monitoring en temps réel
    \item Déploiement sur infrastructure cloud scalable
\end{enumerate}

% ============================================
% RÉFÉRENCES
% ============================================
\begin{thebibliography}{99}

\bibitem{mullin2001}
J.W. Mullin, \textit{Crystallization}, 4th Edition, Butterworth-Heinemann, 2001.

\bibitem{perry2008}
R.H. Perry, D.W. Green, \textit{Perry's Chemical Engineers' Handbook}, 8th Edition, McGraw-Hill, 2008.

\bibitem{randolph1988}
A.D. Randolph, M.A. Larson, \textit{Theory of Particulate Processes}, 2nd Edition, Academic Press, 1988.

\bibitem{streamlit2024}
Streamlit Documentation, \url{https://docs.streamlit.io/}, consulté en 2024.

\bibitem{github2024}
Repository GitHub du projet, \\
\url{https://github.com/lahcenoumoulid3-byte/Projet_evaporation_cristallisation_PIC11.git}

\end{thebibliography}

\end{document}
